\documentclass[10pt,a4paper]{article}

\usepackage{fontspec}
\usepackage{subfigure}
\usepackage[top=1cm]{geometry}
\usepackage{amsmath}
\defaultfontfeatures{Mapping=tex-text}
\setromanfont[Ligatures={Common},Numbers={Lining}]{Linux Libertine}
\author{Michal Staniaszek}
\title{DD2427 Homework 6}

\begin{document}
\maketitle

\section{Eigenface Computation}
We compute eigenfaces differently depending on the size of the matrix $X_c$ which
contains our training data, which is a $d\times n$ matrix. The computation of
the covariance $C$ is 
\begin{equation}
  \label{eq:1}
  C=\frac{1}{n}X_cX_c^T
\end{equation}
This is then a $d\times d$ matrix --- computing the eigenvalues is extremely
expensive. Instead, we compute
\begin{equation}
  \label{eq:2}
  C_1=\frac{1}{n}X_c^TX_c
\end{equation}
and use 
\begin{equation}
  \label{eq:3}
  v_1=X_cv
\end{equation}
to compute an eigenvector $v_1$ of $C$ using the eigenvector $v$ of $C_1$. Both
of these have eigenvalues of $\lambda$. This multiplication results in $d\times
1$ vector, which we would expect for an eigenvector of $C$. It can be shown that
these eigenvectors are identical by the following method. First, we know that an
eigenvector of $C$ looks like
\begin{equation}
  X_cX_c^Tv_1=\lambda v_1\\
\end{equation}
The eigenvectors of $C_1$ look like
\begin{equation}
  \label{eq:4}
  X_c^TX_cv=\lambda v\\
\end{equation}
If we then pre-multiply both sides of \eqref{eq:4} by $X_c$ we get
\begin{equation}
  \label{eq:5}
  X_cX_c^TX_cv=\lambda X_cv\\
\end{equation}
Then, grouping the terms,
\begin{align}
  \label{eq:5}
  X_cX_c^T\left(X_cv\right)=\lambda \left(X_cv\right)\\
  C\left(X_cv\right)=\lambda \left(X_cv\right)
\end{align}
Thus, $X_cv$ must be an eigenvector of $C$ which has the same eigenvalue as the
eigenvector of $C_1$.
  

\section{Eigenfaces and Reconstruction}
\begin{figure}
  \centering
  \subfigure[Bush Dataset]{\includegraphics[width=\linewidth]{results/busheig}}
  \subfigure[ADAFACES Dataset]{\includegraphics[width=\linewidth]{results/adaeig}}
  \caption{First 10 eigenfaces for the Bush and ADAFACES data sets, computed
    using 530 and 3000 images respectively.}
  \label{fig:eigf}
\end{figure}

\begin{figure}
  \centering
  \subfigure[Bush dataset]{\includegraphics[width=0.48\linewidth]{results/bushcumsum}}
  \subfigure[ADAFACES dataset]{\includegraphics[width=0.48\linewidth]{results/adacumsum}}
    \caption{Cumulative sum of the eigenvalues for the Bush and ADAFACES sets
      using 530 and 3000 images respectively. The red line indicates the number
      of eigenvectors required to retain 90\% of the variation in the data.}
  \label{fig:csum}
\end{figure}

\begin{figure}
  \centering
  \subfigure[Original]{\includegraphics[width=0.2\linewidth]{results/student/orig}}
  \subfigure[1]{\includegraphics[width=0.2\linewidth]{results/student/rec1}}
  \subfigure[10]{\includegraphics[width=0.2\linewidth]{results/student/rec10}}
  \subfigure[20]{\includegraphics[width=0.2\linewidth]{results/student/rec20}}\\
  \subfigure[50]{\includegraphics[width=0.2\linewidth]{results/student/rec50}}
  \subfigure[100]{\includegraphics[width=0.2\linewidth]{results/student/rec100}}
  \subfigure[200]{\includegraphics[width=0.2\linewidth]{results/student/rec200}}
  \subfigure[250]{\includegraphics[width=0.2\linewidth]{results/student/rec250}}
  \caption{Reconstruction of a face using different numbers of eigenfaces
    computed from the 3000 examples in the ADAFACES set.}
  \label{fig:rec}
\end{figure}

\end{document}
